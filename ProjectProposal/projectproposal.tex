\documentclass[12pt,serif,draftclsnofoot,onecolumn]{Article}
\usepackage{color}
\usepackage{setspace}
\usepackage{url}
\singlespacing

\newcommand{\HRule}[1]{\rule{\linewidth}{#1}}
\begin{document}
	\begin{titlepage}


	\title{ \normalsize \textsc{}
			\\ [2.0cm]
			\HRule{0.5pt} \\
			\LARGE \textbf{\uppercase{Project Proposal}}
			\\ \normalsize \textsc{Accurate Earth Orbit}
			\HRule{2pt} \\ [0.5cm]
			\normalsize \today \vspace*{5\baselineskip}}
	\date{10/16/2016}
	
	\author{Steven Silvers \\
			silverss@oregonstate.edu \\
			Oregon State University \\
			CS 450 Intro to Computer Graphics}
	\pagenumbering{gobble}	
	\maketitle
	\end{titlepage}
	\newpage
	\pagenumbering{arabic}
	
	\par
			While similar to the popular solar system project, this project will only involve the Earth's orbit around the sun and the moon's orbit around the earth. The Earth and moon orbits will both be to scale and follow their accurate elliptical paths, which will be a difficult transformation challenge. Getting an accurate speed of the orbit is difficult enough with a circular orbit, so going on an elliptical path provides an even greater challenge. The planets will also be transformed to rotate on their axes as they would in real life.
	\newline
	\par
			Textures will be applied to the Sun, Earth and moon to give the model a realistic look. The only source of light will be the sun to make this model as accurate as possible. I plan to be able to show eclipses with this model to add difficulty to the lighting aspect of the project.

\end{document}